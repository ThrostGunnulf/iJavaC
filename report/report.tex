% !TeX spellcheck = pt_PT
\documentclass[11pt,a4paper]{article}
\usepackage{etex}
\reserveinserts{28}

\usepackage[utf8]{inputenc}

\usepackage{hyperref}
\usepackage{amsthm}
\usepackage{amsmath}
\usepackage{amssymb}
\usepackage{amsfonts}
\usepackage[portuguese]{babel}
\usepackage{pictex}
\usepackage[dvips]{graphics}
\usepackage{enumerate}
\usepackage{color}
\usepackage[usenames,dvipsnames]{xcolor}

\usepackage{amsmath}
\usepackage{amssymb}
\usepackage{amsfonts}
\usepackage[portuguese]{babel}
\usepackage{pictex}
\usepackage[dvips]{graphics}
\usepackage{enumerate}
\usepackage{color}
\usepackage{indentfirst}
\usepackage{listings}
\usepackage{tabularx}
\usepackage{multirow}
\usepackage{float}
\usepackage{makeidx}

\usepackage{graphicx}
\setlength{\parindent}{1cm}


%----------------------------------------------------------------------------------------
%	TITLE PAGE
%----------------------------------------------------------------------------------------

\newcommand*{\titleGM}{\begingroup % Create the command for including the title page in the document
\hbox{ % Horizontal box
\hspace*{0.2\textwidth} % Whitespace to the left of the title page
\rule{1pt}{\textheight} % Vertical line
\hspace*{0.05\textwidth} % Whitespace between the vertical line and title page text
\parbox[b]{0.75\textwidth}{ % Paragraph box which restricts text to less than the width of the page

{\noindent\Huge\bfseries Compilador \\[0.5\baselineskip] iJava}\\[2\baselineskip] % Title
{\large \textit{Projecto da Cadeira de Compiladores}}\\[4\baselineskip] % Tagline or further description
{\Large \textsc{JOÃO OLIVEIRA - 2010129856}} \\ \\ % Author name
{\Large \textsc{JOÃO SIMOES - 2011150045}} % Author name

\vspace{0.5\textheight} % Whitespace between the title block and the publisher
{\noindent FCTUC - Departamento de Engenharia Informática}\\[\baselineskip] % Publisher and logo
}}
\endgroup}

\begin{document}

\titleGM

\section{Inrodução}

Este projecto consistiu no desenvolvimento de um compilado para a linguagem $iJava$. Esta línguagem apresenta diversas caracteristicas específicas a serem descritas de seguida.

Uma das características prende-se com o facto de cada programa conter uma única classe. Esta classe pode ter incluídas variáveis estáticas e métodos \emph{public static}, assim como variáveis locais. Para além disto, é mandatória a presença de um método \emph{main}, método este a ser chamado aquando da execução do programa. Este mesmo método tem como parâmetro, por defeito, um array de \emph{Strings}, contendo os parâmetos de entrada.

É ainda possível implementar expressões aritméticas e lógicas, assim como operações de atribuição, relacionais e controlo. (\emph{if-else} e \emph{while} )

Relativamente aos tipos de variáveis existentes na linguagem, é possivel usar varíaveis do tipo \emph{Integer} e \emph{Boolean}. É ainda possível criar arrays unidimensionais dos tipo apresentados anteriormente.

Por fim, o desenvolvimento do compilador deu-se em várias fases bem definidas, fases estas que apresentamos de seguida.

\begin{description}
	\item[Fase 1] Análise Lexical
		\begin{itemize}
			\item Identificação dos tokens aceites pela linguagem e tradução dos mesmo para a linguagem \emph{LEX}. 
		\end{itemize}
	\item[Fase 2] Análise Sintática
		\begin{itemize}
			\item Tradução da gramática dada para a linguagem \emph{YACC} e resolução de conflitos e ambiguidades;
			\item Estruturas de dados a serem criadas de forma a representar os nós da Árvore de Sintaxe Abstracta;
			\item Definição de enumerações de forma a representar os diversos tipos de operações,  expressões, \emph{statments}, entre outros;
			\item Implementação das funções responsáveis pela construção da AST;
			\item Deteção de erros sintáticos.
		\end{itemize}
	\item[Fase 3] Análise Semântica
		\begin{itemize}
			\item Definição das estruturas de dados a serem utilizadas para a criação da tabela de símbolos;
			\item Implementação dos procedimentos responsáveis pelas inserções na tabela de símbolos;
			\item Detecção de erros ao nível semântico. 
		\end{itemize}
	\item[Fase 4] Geração de código intermédio
		\begin{itemize}
			\item Implemententação das funções responsáveis pela geração de código \emph{LLVM}.
		\end{itemize}
\end{description}

\newpage

\section{Análise Léxical}

A primeira fase da construção do compilador consistiu na definição dos tokens aceites pela linguagem, através de expressões regulares, sendo internamente construído pelo \emph{LEX} um autómato.


   




\end{document}