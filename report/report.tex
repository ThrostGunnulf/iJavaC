% !TeX spellcheck = pt_PT
\documentclass[11pt,a4paper]{article}

\usepackage[margin=1.3in]{geometry}
\usepackage[portuguese]{babel}
\usepackage[utf8]{inputenc}
\usepackage{indentfirst}
\usepackage{graphicx}
\usepackage{listings}


\setlength{\parindent}{1cm}


%----------------------------------------------------------------------------------------
%	TITLE PAGE
%----------------------------------------------------------------------------------------

\newcommand*{\titleGM}{\begingroup % Create the command for including the title page in the document
\hbox{ % Horizontal box
\hspace*{0.2\textwidth} % Whitespace to the left of the title page
\rule{1pt}{\textheight} % Vertical line
\hspace*{0.05\textwidth} % Whitespace between the vertical line and title page text
\parbox[b]{0.75\textwidth}{ % Paragraph box which restricts text to less than the width of the page

{\noindent\Huge\bfseries Compilador \\[0.5\baselineskip] iJava}\\[2\baselineskip] % Title
{\large \textit{Projecto da Cadeira de Compiladores}}\\[4\baselineskip] % Tagline or further description
{\Large \textsc{João Oliveira - 2010129856}} \\ \\ % Author name
{\Large \textsc{João Simões - 2011150045}} % Author name

\vspace{0.5\textheight} % Whitespace between the title block and the publisher
{\noindent FCTUC - Departamento de Engenharia Informática}\\[\baselineskip] % Publisher and logo
}}
\endgroup}

\begin{document}

\titleGM

\section{Introdução}

Este projecto consistiu no desenvolvimento de um compilador para a linguagem \textit{iJava}, de imperative \textit{Java}. Esta linguagem é uma restrição da linguagem \textit{Java}, para facilitar a implementação do compilador.

Uma das características da linguagem é o facto de cada ficheiro conter uma única classe, sendo que cada programa consiste apenas num ficheiro. Uma classe pode ter variáveis globais e métodos, que por sua vez podem ter variáveis locais. Para além disto, é obrigatória a existência de um método \textit{main}, método este que será a função de entrada do programa. Este mesmo método tem como parâmetro, por defeito, um array de \textit{Strings}, contendo os parâmetos de entrada, tal como em \textit{Java}.

É ainda possível implementar expressões aritméticas, lógicas e relacionais, assim como statements de atribuição, indexação e controlo (\textit{if-else} e \textit{while}).

Relativamente aos tipos de variáveis existentes na linguagem, é possivel usar varíaveis do tipo \textit{int} e \textit{boolean}. É ainda possível criar arrays unidimensionais dos tipo apresentados anteriormente.

Por fim, o desenvolvimento do compilador fez-se em várias fases bem definidas, fases essas que apresentamos de seguida.

\begin{description}
	\item[Fase 1] Análise Lexical
		\begin{itemize}
			\item Identificação dos tokens aceites pela linguagem, recorrendo ao \textit{LEX}. Nesta ferramenta definimos os tokens através de expressões regulares. 
		\end{itemize}
	\item[Fase 2] Análise Sintática
		\begin{itemize}
			\item Tradução da gramática dada para a linguagem \textit{YACC} e resolução de conflitos e ambiguidades;
			\item Criação das estruturas de dados que representam os nós da Árvore de Sintaxe Abstracta;
			\item Implementação das funções responsáveis pela construção da AST;
			\item Deteção de erros sintáticos.
		\end{itemize}
	\item[Fase 3] Análise Semântica
		\begin{itemize}
			\item Criação das estruturas de dados a utilizar para a construção da tabela de símbolos;
			\item Implementação dos procedimentos responsáveis pelas inserções na tabela de símbolos;
			\item Detecção de erros semânticos.
		\end{itemize}
	\item[Fase 4] Geração de código intermédio
		\begin{itemize}
			\item Implemententação das funções responsáveis pela geração de código \textit{LLVM}.
		\end{itemize}
\end{description}

\newpage

\section{Análise Léxical}

A primeira fase da construção do compilador consistiu na definição dos tokens aceites pela linguagem, através de expressões regulares. Para este efeito usou-se o \textit{LEX}, que constrói internamente um autómato determinístico capaz de reconhecer esses tokens.

\subsection{Tokens}

Após terminada a fase de análise lexical, obtivemos os seguintes \textit{tokens}:

\begin{itemize}
	\item \textbf{INT} - \textit{int} 
	\item \textbf{BOOL} - \textit{boolean}
	\item \textbf{NEW} - \textit{new}
	\item \textbf{IF} - \textit{if}
	\item \textbf{ELSE} - \textit{else}
	\item \textbf{WHILE} - \textit{while}
	\item \textbf{PRINT} - \textit{System.out.println}
	\item \textbf{PARSEINT} - \textit{Integer.parseInt}
	\item \textbf{CLASS} - \textit{class}
	\item \textbf{PUBLIC} - \textit{public}
	\item \textbf{STATIC} - \textit{static}
	\item \textbf{VOID} - \textit{void}
	\item \textbf{STRING} - \textit{String}
	\item \textbf{DOTLENGTH} - \textit{.length}
	\item \textbf{RETURN} - \textit{return}
	\item \textbf{OCURV} - $($
	\item \textbf{CCURV} - $)$
	\item \textbf{OBRACE} - \{
	\item \textbf{CBRACE} - \}
	\item \textbf{OSQUARE} - [
	\item \textbf{CSQUARE} - ]
	\item \textbf{OP1} - \&\& , $\|$
	\item \textbf{OP2} - $< , > , == , != , <= , >=$
	\item \textbf{OP3} - $+ , -$
	\item \textbf{OP4} - $* , / , \% $
	\item \textbf{NOT} - !
	\item \textbf{ASSIGN} - $=$
	\item \textbf{SEMIC} - ;
	\item \textbf{COMMA} - ,
	\item \textbf{BOOLLIT} - \textit{true, false}
	\item \textbf{ID} - Corresponde a todas as sequências alfanuméricas inciadas por uma letra (maiúscula ou minúscula), que podem conter os caracteres "\$" e "$\_$".
\end{itemize}

\subsection{Detecção de Erros Lexicais}

A análise lexical é realizada da esquerda para a direita, caracter a caracter, até ao momento em que o autómato atinge um estado morto. Quando este mesmo estado é atingido, o analisador regressa ao último estado final, caso em que é encontrado um \textit{token} válido. Caso o analisador não tenha atingido previamente um estado final, é então gerado um erro lexical. Para melhor identificação do erro lexical, é impressa a linha e a coluna da posição onde começa o \textit{token}. Para este efeito, necessitamos de mecanismos para obter informação da posição.

Para tal, existe uma variável com o nome \textit{colNo}, responsável por armazenar a coluna actual, sendo esta variável incrementada de cada vez que se acaba de processar um \textit{token} válido. Relativamente à linha, obtemos o seu valor actual através da variável \textit{yylineno}, já implementada pelo \textit{YACC}.


\subsection{Tratamento dos Comentários}

A linguagem \textit{iJava} permite a existência de comentários de linha (iniciados por "\textit{//}") e comentários de bloco (colocados entre "\textit{/* */}").

No caso dos comentários de linha, uma simples expressão regular permite manter actualizada a informação de linhas e colunas. No caso dos comentários de bloco, usou-se um estado \textit{COMMENT}, no qual se entra quando se detecta o \textit{token} "\textit{/*}" e do qual se sai quando se detecta o \textit{token} "\textit{*/}". Dentro deste estado, apesar de se ignorar o texto que não representa o final do comentário, actualiza-se a informação de posição, mantendo assim a sua correcção.

\newpage

\section{Análise Sintática}

Após terminada a fase da análise lexical, segue-se a análise sintática. Uma vez que nesta fase é necessário ter em conta as prioridades dos operadores, foram feitos alguns ajustes nos tokens detectados. Estas alterações são explicadas na próxima secção.

A ferramenta utilizada, o \textit{YACC}, é um analizador sintático \textit{LALR(1)}. Assim, foi preferida na nossa implementação da gramática recursividade à esquerda, por resultar numa menor utilização da pilha.

\subsection{Alterações na Análise Lexical}

Como dito anteriormente, houve necessidade de alterar a organização dos \textit{tokens} relativos aos operadores. Estas alterações devem-se à necessidade de se ter em conta as prioridades dos operadores. Assim, após alterações, obtivemos os seguintes tokens relativos aos operadores.

\begin{itemize}
	\item \textbf{AND} - \&\&
	\item \textbf{OR} - $\|$
	\item \textbf{RELCOMPAR} - $< , >, <= , >=$
	\item \textbf{EQUALITY} -  $== , !=$ 
	\item \textbf{ADDITIVE} - $+ , -$
	\item \textbf{MULTIPLIC} - $* , / , \% $
\end{itemize}

Para além das alterações acima referidas, foram ainda feitas outras em relação à contagem da linha e da coluna. Uma vez que aquando da mostra da mensagem de erro sintático, o número da linha e da coluna apresentadas têm de corresponder ao início do \textit{token} inválido, esta informação foi guardada de forma a poder ser acedida no analisador sintático.

\subsection{Gramática da Linguagem}

A gramática da linguagem \textit{iJava}, apresenta-se de seguida segundo a notação \textit{EBNF}: 
\vspace{0.5cm}

\hspace{-1cm}\vspace{.1cm}Start $\rightarrow$ Program \\ \\
\vspace{.5cm}Program $\rightarrow$ CLASS ID OBRACE { FieldDecl $\mid$ MethodDecl } CBRACE\\
\vspace{.5cm}FieldDecl $\rightarrow$ STATIC VarDecl\\
MethodDecl $\rightarrow$ PUBLIC STATIC ( Type $\mid$ VOID ) ID OCURV [ FormalParams ] CCURV OBRACE { VarDecl } { Statement } CBRACE\\ \\
\vspace{.5cm}FormalParams $\rightarrow$ Type ID { COMMA Type ID }\\
\vspace{.5cm}FormalParams $\rightarrow$ STRING OSQUARE CSQUARE ID\\
\vspace{.5cm}VarDecl $\rightarrow$ Type ID { COMMA ID } SEMIC\\
\vspace{.5cm}Type $\rightarrow$ ( INT $\mid$ BOOL ) [ OSQUARE CSQUARE ]\\
\vspace{.5cm}Statement $\rightarrow$ OBRACE { Statement } CBRACE\\
\vspace{.5cm}Statement $\rightarrow$ IF OCURV Expr CCURV Statement [ ELSE Statement ]\\
\vspace{.5cm}Statement $\rightarrow$ WHILE OCURV Expr CCURV Statement\\
\vspace{.5cm}Statement $\rightarrow$ PRINT OCURV Expr CCURV SEMIC\\
\vspace{.5cm}Statement $\rightarrow$ ID [ OSQUARE Expr CSQUARE ] ASSIGN Expr SEMIC\\
\vspace{.5cm}Statement $\rightarrow$ RETURN [ Expr ] SEMIC\\
\vspace{.5cm}Expr $\rightarrow$ Expr ( OP1 $\mid$ OP2 $\mid$ OP3 $\mid$ OP4 ) Expr\\
\vspace{.5cm}Expr $\rightarrow$ Expr OSQUARE Expr CSQUARE\\
\vspace{.5cm}Expr $\rightarrow$ ID $\mid$ INTLIT $\mid$ BOOLLIT\\
\vspace{.5cm}Expr $\rightarrow$ NEW ( INT $\mid$ BOOL ) OSQUARE Expr CSQUARE\\
\vspace{.5cm}Expr $\rightarrow$ OCURV Expr CCURV\\
\vspace{.5cm}Expr $\rightarrow$ Expr DOTLENGTH $\mid$ ( OP3 $\mid$ NOT ) Expr\\
\vspace{.5cm}Expr $\rightarrow$ PARSEINT OCURV ID OSQUARE Expr CSQUARE CCURV\\
\vspace{.5cm}Expr $\rightarrow$ ID OCURV [ Args ] CCURV\\
\vspace{.5cm}Args $\rightarrow$ Expr { COMMA Expr }\\


Após a interpretação da gramática dada, definimos a gramática no \textit{YACC}. A gramática obtida após definição de prioridades e após alterações relativamente à gramática acima, foi a seguinte (todas as alterações e definições de precedências serão abordadas com detalhe nas proximas secções).

\vspace{0.7cm}

\lstset{language=C,caption={Gramática obtida segundo a representação do YACC},label=Estruturas1,numbers=left,frame=single, breaklines = true}
\begin{lstlisting}

start: CLASS ID '{' decls '}'
     | CLASS ID '{' '}'

decls: decls fielddecl
     | decls methoddecl

fielddecl: STATIC type ID idlist ';'

methoddecl: PUBLIC STATIC methodtype ID '(' formalparams ')' '{' vardecl stmtlist '}'

methodtype: type 
          | VOID
          
formalparams: type ID formalparamslist
            | STRING '[' ']' ID
            |

formalparamslist: formalparamslist ',' type ID   
                |

stmtlist: stmtlist statement
        |

vardecl: vardecl type ID idlist ';'
       |

idlist: idlist ',' ID
      |

type: INT '[' ']'
    | BOOL '[' ']'
    | INT
    | BOOL

statement: '{' stmtlist '}'
         | IF '(' expr ')' statement ELSE statement  %prec ELSE
         | IF '(' expr ')' statement    %prec IFX   
         | WHILE '(' expr ')' statement
         | PRINT '(' expr ')' ';'
         | ID '[' expr ']' '=' expr ';' 
         | ID '=' expr ';'
         | RETURN expr ';'
         | RETURN ';'         

expr: exprindex
    | exprnotindex
	
exprindex: ID
         | INTLIT
         | BOOLLIT
         | '(' expr ')'
         | expr DOTLENGTH
         | PARSEINT '(' ID '[' expr ']' ')'
         | ID '(' args ')'
         | ID '(' ')'
         | exprindex '[' expr ']'

exprnotindex: NEW INT '[' expr ']'
            | NEW BOOL '[' expr ']'
            | '!' expr               %prec UNARY
            | ADDITIVE expr          %prec UNARY
            | expr AND expr
            | expr OR expr
            | expr RELCOMPAR expr
            | expr EQUALITY expr
            | expr ADDITIVE expr
            | expr MULTIPLIC expr

args: expr argslist
    | expr
 
argslist: ',' args

\end{lstlisting}

\subsection{Tradução da Gramática dada para o YACC}

Como referido anteriormente, foram necessárias diversas alterações à gramática dada na notação \textit{EBNF}. Nesta sub-secção vamos comentar essas alterações explicando a sua razão de ser.

Uma das alterações efectuadas, é relativa à abordagem do que é "opcional"$([\dots])$, que tem "zero ou mais repetições"$(\{\dots\})$ e situações em que temos várias opções.

\begin{itemize}
	\item Nas situações em que temos símbolos "opcionais", dividimos a regra em duas regras distintas para abrager os dois casos posssíveis, o caso em que tem o símbolo e o caso em que não. Noutros casos, simplesmente considerámos a possiblidade de ter um síbolo não terminal a derivar a cadeia vazia.
	
	\item Quando existem "zero ou mais repetições", considerámos a possibilidade de ser derivada a cadeia vazia, sendo ainda adicionada recursividade de forma a permitir várias repetições do mesmo símbolo.
	
	\item No caso de termos várias opções relativamente ao símbolo, é criada uma nova regra que contempla todos os símbolos possíveis.  
\end{itemize} 

\subsubsection{Prioridade de Operadores}

Outras alterações efectuadas têm a ver com a definição de prioridade de operadores. Apresentamos abaixo as prioridades que definimos no \textit{YACC}.

\newpage 
 
\lstset{language=C,caption={Prioridades},label=Estruturas2,frame=single, breaklines = true}
\begin{lstlisting}
 %left OR
 %left AND
 %left EQUALITY
 %left RELCOMPAR
 %left ADDITIVE
 %left MULTIPLIC
 %right UNARY
 %left '[' DOTLENGTH
\end{lstlisting}

\vspace{0.5cm}

Segundo o \textit{YACC}, as prioridades definidas mais abaixo, têm maior prioridade do que as definidas acima. Por exemplo, \textit{AND} tem maior prioridade do que \textit{OR} e menor do que \textit{EQUALITY}. 

Estas prioridades representam as prioridades da linguagem \textit{Java}, que são as mesmas que se aplicam na linguagem \textit{iJava}.

\subsubsection{Ambiguidade \textit{if-else}}

Uma ambiguidade muito comum na definição da gramática de uma linguagem está relacionada com o \textit{if-else}. Veja-se o seguinte exemplo:

\vspace{0.3cm}
\lstset{language=C,label=Estruturas3, caption={}, numbers=none, frame=single, breaklines = true}
\begin{lstlisting}
 IF '(' expr ')' IF '(' expr ')' statement ELSE statement  
\end{lstlisting}
 
\vspace{0.3cm}

Como podemos verficar acima, o " \textit{ELSE statement} " pode estar associado ao \textit{IF} exterior ou ao interior. São então possíveis duas árvores de derivação, existindo portanto uma ambiguidade. Para resolver esta ambiguidade, foram definidas as seguintes prioridades:

\vspace{0.3cm}

\lstset{language=C,label=Estruturas4, caption={}, numbers=none, frame=single, breaklines = true}
\begin{lstlisting}
 %nonassoc IFX
 %nonassoc ELSE   
\end{lstlisting}

\vspace{0.3cm}

Desta forma, dando maior prioridade à redução do \textit{ELSE}, o \textit{ELSE} é sempre associado ao \textit{IF} mais recente.

\subsubsection{Indexação}

Relativamente à indexação, foram feitas algumas alterações relativamente à gramática inicial, de forma separar as expressões indexáveis das não-indexáveis. Como podemos verificar (ver linhas 46-68), a \textit{expr} pode derivar em \textit{exprindex} e \textit{exprnotindex}.

\begin{itemize}
	\item Uma vez que a linguagem não permite a existência de arrays \textbf{não unidimensionais}, as regras " \textit{NEW INT [ expr ]} " e " \textit{NEW BOOL [ expr ]} " estão incluídas nas não indexáveis, não sendo assim possível a inicialização de arrays não unidimensionais.
	
	\item O operador de indexação tem maior prioridade do que qualquer operador (excepto \textit{DOTLENGTH}, que tem a mesma). Desta forma, nunca será possível indexar uma expressão unária (excepto \textit{DOTLENGTH}) ou binária, pelo que estas são não-indexáveis. 
	
	\item Todos os outros tipo de expressões, estão incluídas nas indexáveis.
\end{itemize}

\subsection{Árvore de Sintaxe Abstracta}

Concluída a definição da grámatica, segue-se então a construção da AST.

\subsubsection{Estruturas de Dados}

Para a construção da AST, foram definidos diversos nós, sendo estes apresentados e analisados detalhadamente de seguida. Optámos por definir um nó para cada regra da gramática, por ser para nós conceptualmente mais fácil visualizar a árvore desta forma.

\lstset{caption={Expressão}}
\begin{lstlisting}
 typedef struct _expr
 {
     ExprType type;
     OpType op;
 	struct _expr *expr1;
 	struct _expr *expr2;
 	char *idOrLit;
 	ArgsList *argsList;
 } Expr;
\end{lstlisting}

\vspace{0.3cm}

\begin{itemize}
	\item \textbf{type} - permite identificar o tipo da expressão (Binária, Unária, \textit{ID}, \textit{boolean}, \textit{int}, Chamada de Função, \textit{parseInt}, indexação, \textit{new BOOL[]}, \textit{new INT[]});
	
	\item \textbf{op} - permite identificar o operador aplicado na expressão. Estes operadores são aplicados nas expressões binárias, unárias e no caso particular do \textit{.length};
	
	\item \textbf{expr1} - esta variável corresponde à expressão à esquerda do operador nas operações binárias, assim como a expressão utilizada nas operações unárias. No caso da indexação, este campo corresponde à expressão a ser indexada. Por fim, no caso das operações do tipo \textit{new BOOL[]}, \textit{new INT[]}, este campo corresponde ao tamanho do array a ser inicializado.
	
	\item \textbf{expr2} - no caso das operações binárias, a \textit{expr2} corresponde à expressão à direita do operador. Este campo é utilizada nas expressões do tipo \textit{Indexação}, correspondendo ao índice a ser usado.
	
	\item \textbf{idOrLit} - esta variável tem como objectivo armazenar um \textit{ID} da expressão ou um literal;
	
	\item \textbf{argsList} - é através desta variável que são guardados os argumentos "passados" quando é feita a chamada de uma função.
\end{itemize}

\newpage

\lstset{caption={Lista de Argumentos}}
\begin{lstlisting}
 typedef struct _argsList ArgsList;

 struct _argsList
 {
 	Expr *expr;
     struct _argsList *next;
 };
\end{lstlisting}

\vspace{0.3cm}

Representa a lista de argumentos a serem passados aquando da chamada de uma função.

\lstset{caption={Statement}}
\begin{lstlisting}
 typedef struct _stmt
 {
     StmtType type;
     char *id;
     Expr *expr1;
     Expr *expr2;
     struct _stmt *stmt1;
     struct _stmt *stmt2;
     struct _stmtList *stmtList;
 } Stmt;
\end{lstlisting}

\vspace{0.3cm}

\begin{itemize}
    \item \textbf{type} - permite distinguir os vários tipos de statements (\textit{compound statement}, \textit{if-else}, \textit{return}, \textit{while}, \textit{print}, \textit{store}, \textit{store array});
    
    \item \textbf{id} - esta variável permite guardar o \emph{id} da variável a que é atribuida uma expressão, no \emph{store} e no \emph{storearray};
    
    \item \textbf{expr1} - é utilizada no caso de existir apenas uma expressão na regra. No caso de existirem duas regras, a regra mais à exquerda é armazenada neste campo.
    
    \item \textbf{expr2} - apenas é utilizada no caso de existirem duas expressões na regra, correspondendo à expressão mais à direita;
    
    \item \textbf{stmtList} - caso o \emph{statement} seja do tipo \emph{compound statement}, então todas as \emph{statements} são adicionadas a esta lista de \emph{statments}.
\end{itemize}

\lstset{caption={Lista de Statements}}
\begin{lstlisting}
 typedef struct _stmtList
 {
     Stmt *stmt;
     struct _stmtList *next;
 } StmtList;
\end{lstlisting}

\vspace{0.3cm}

Esta lista é utilizada como estrutura auxiliar de \textit{compound statements}.

\lstset{caption={Lista de IDs}}
\begin{lstlisting}
 typedef struct _idList
 {
 	char *id;
 	struct _idList *next;
 } IDList;
\end{lstlisting}

\vspace{0.3cm}

Permite armazenar os id's das variáveis quando estas são declaradas em formato de lista definindo o tipo delas uma única vez.

\lstset{caption={Declaração de Variável}}
\begin{lstlisting}
 typedef struct _varDecl
 {
     Type type;
	 int isStatic;
	 IDList *idList;
 } VarDecl;
\end{lstlisting}

\vspace{0.3cm}

\begin{itemize}
    \item \textbf{type} - corresponde ao tipo da variável a ser declarada(\textit{int, bool, int[], bool[]});
    
    \item \textbf{isStatic} - identifica a variável como sendo estática ou não;
    
    \item \textbf{idList} - uma vez que podem ser declaradas várias variáveis do mesmo tipo definindo o tipo apenas uma vez, é usada uma estrutura auxiliar para manter a lista dos \textit{ID}s das variáveis.
\end{itemize}

\lstset{caption={Lista de Declarações}}
\begin{lstlisting}
 typedef struct _varDeclList
 {
  	 VarDecl *varDecl;
	 struct _varDeclList *next; 
 } VarDeclList;
\end{lstlisting}

Estrutura auxiliar usada para manter todas as declarações de variáveis locais de um método.

\vspace{0.3cm}

\lstset{caption={Lista de Parâmetros}}
\begin{lstlisting}
 typedef struct _paramList
 {
     Type type;
 	char *id;
 	struct _paramList *next;
 } ParamList;
\end{lstlisting}

\vspace{0.3cm}

\begin{itemize}
    \item \textbf{type} - representa o tipo do parâmetro;
    \item \textbf{id} - armazena o \textit{id} do parâmetro;
    \item \textbf{next} - ponteiro para o parâmetro seguinte.
\end{itemize}

\lstset{caption={Declaração de Método}}
\begin{lstlisting}
 typedef struct _methodDecl
 {
     Type type;
 	char *id;
 	ParamList *paramList;
 	VarDeclList *varDeclList;
 	StmtList *stmtList;
 } MethodDecl;
\end{lstlisting}

\vspace{0.3cm}

\begin{itemize}
    \item \textbf{type} - representa o tipo do valor de retorno da função. Para além dos tipos de variáveis é ainda possível retornar \textit{void};
    \item \textbf{id} - armazena o nome do método;
    \item \textbf{paramList} - contém a lista de parâmetros recebidos pela função;
    \item \textbf{varDeclList} - lista de declarações de variáveis locais à função;
    \item \textbf{stmtList} - lista de \textit{statments da função}.
\end{itemize}

\lstset{caption={Lista de Declarações}}
\begin{lstlisting}
 typedef struct _declList
 {
     DeclType type;
 	union
 	{
 		VarDecl *varDecl;
 		MethodDecl *methodDecl;
 	};
     struct _declList *next;
 } DeclList;
\end{lstlisting}

\vspace{0.3cm}

\begin{itemize}
    \item \textbf{type} - permite identificar se é a declaração de um método ou de uma variável global;
    \item \textbf{varDecl} - caso se trate de uma variável global, então é criado um novo nó do tipo \textit{Variable Declaration};
    \item \textbf{methodDecl} - caso se trate da declaração de um método, então é criado um novo nó do tipo \textit{Method Declaration}
\end{itemize}

\lstset{caption={Classe}}
\begin{lstlisting}
 typedef struct _class
 {
 	char *id;
 	DeclList *declList;
 } Class;
\end{lstlisting}

\vspace{0.3cm}

Esta estrutura armazena o nome da classe e a lista de declarações.

\subsubsection{Construção da Árvore}

Após termos definido as estrutura de dados acima, seguem-se os procedimentos utilizados para efectuar novas inserções na AST.

\lstset{caption={Funções utilizadas na construção da AST}}
\begin{lstlisting}
 Class* insertClass(char*, DeclList*);
 
 DeclList* insertDecl(DeclType, void*, DeclList*);
 
 VarDecl* insertFieldDecl(Type, char*, IDList*);
 
 VarDeclList* insertVarDecl(VarDeclList*, Type, char*, IDList*);
 
 IDList* insertID(char*, IDList*);
 
 StmtList* insertStmtList(Stmt*, StmtList*);
 
 Stmt* insertStmt(StmtType, char*, Expr*, Expr*, Stmt*, Stmt*,   StmtList*);
 
 ParamList* insertFormalParam(Type, char*, ParamList*, int);
 
 MethodDecl* insertMethodDecl(Type, char*, ParamList*, VarDeclList*, StmtList*);
 
 Expr* insertExpr(ExprType, char*, Expr*, Expr*, char*, ArgsList*);
 
 ArgsList* insertArg(Expr*, ArgsList*);
\end{lstlisting}
\vspace{0.3cm}

\section{Análise Semântica}

Concluída a construção da \textit{AST} e a identificação de erros sintáticos, segue-se a análise semântica. Para tal, foi construída uma tabela de símbolos global e uma tabela local para cada método declarado.

\subsection{Arquitetura da Tabela de Símbolos}

Segue-se abaixo um esquema da estrutura da tabela de simbolos.

\begin{figure}[h!]
    \centering
    \centerline{\includegraphics[keepaspectratio=true, width=1.3\textwidth]{table.jpg}}
    \caption{Esquema tabela de símbolos.}
\end{figure}

Como se pode verificar, a tabela é representada através de uma lista ligada em que os nós representam variáveis globais e métodos. Cada método contém um ponteiro para uma nova lista, a sua tabela local, que contém todos os símbolos locais a esse método. Os argumentos dos métodos aparecem no ínicio desta lista (com a \textit{flag} \textit{isParam} a 1).

De seguida, apresentam-se as estruturas de dados utilizadas para representar a tabela de símbolos.

\lstset{caption={Estrutura da Tabela de Símbolos}}
\begin{lstlisting}
 typedef struct _methodTableEntry
 {
     char* id;
     Type type;
     int isParam;
     struct _methodTableEntry* next;
 } MethodTableEntry;

 typedef struct _methodTable
 {
     struct _classTable* broaderTable;
     MethodTableEntry* entries;

 } MethodTable;

 typedef struct _classTableEntry
 {
     char* id;
     Type type;
     MethodTable* methodTable;
     struct _classTableEntry* next;
 } ClassTableEntry;

 typedef struct _classTable
 {
     char* id;
     ClassTableEntry* entries;
 } ClassTable;
\end{lstlisting}

Os métodos utilizados na construção da tabela de símbolos foram os que se apresentam na listagem seguinte.

\lstset{caption={Métodos para Construção da Tabela de Símbolos}}
\begin{lstlisting}
ClassTable* buildSymbolsTables(Class*);
ClassTableEntry* newVarEntries(VarDecl*, ClassTableEntry*);
void newMethodEntry(MethodDecl*, ClassTableEntry*, ClassTable*);
void newMethodTable(MethodTable*, ParamList*, VarDeclList*);
\end{lstlisting}

\begin{itemize}
    \item O método \textit{buildSymbolsTables()} é o método que é chamado no \textit{YACC} após a construção da \textit{AST} para dar início à construção da tabela de símbolos.
    \item \textit{newVarEntries} é o método que, a partir de uma lista de variáveis globais em que o tipo apenas é definido uma vez, as coloca na tabela de símbolos da classe.
    \item O método \textit{newMethodEntry()} é responsável por introduzir na tabela de símbolos da classe toda a informação referente ao método, como o seu nome e tipo de retorno.
    \item \textit{newMethodTable()} é o método que preenche a tabela de símbolos local a um método com a informação sobre os seus parâmetros e variáveis locais.
\end{itemize}

\subsection{Detecção de Erros}

Após a construção da tabela de símbolos, segue-se a deteção de erros semânticos. Apresentam-se de seguida os possíveis erros semânticos:

\lstset{caption={Estrutura da Tabela de Símbolos}, numbers=left}
\begin{lstlisting}
 Cannot find symbol %s
 Incompatible type of argument %d in call to method %s (got % required %s)
 Incompatible type in assignment to %s (got %s, required %s)
 Incompatible type in assignment to %s[] (got %s, required %s)
 Incompatible type in %s statement (got %s, required %s)
 Incompatible type in %s statement (got %s, required %s or %s)
 Invalid literal %s
 Operator %s cannot be applied to type %s
 Operator %s cannot be applied to types %s, %s
 Symbol %s already defined
\end{lstlisting}

\begin{itemize}
    \item \textbf{Erro 1} - A verificação da declaração de uma variável ou método já definidos é feita durante a construção da tabela de símbolos. Assim, antes da inserção de um novo símbolo na tabela, verifica-se se o símbolo já foi previamente definido. Note-se que, a linguagem permite a existência de um método e de uma várivel local com o mesmo nome, não sendo por isso gerado nenhum erro semântico;
    
    \item \textbf{Erro 2} - A comparação entre os tipos de parâmetros recebidos por um método e os argumentos passados na chamada deste mesmo método, é feita obtendo da tabela de símbolos local ao método os tipos recebidos pela função, comparando-os com os argumentos passados, obtidos da \textit{AST};
    
    \item \textbf{Erro 3, 4} - A verificação da correcção de tipos nas atribuições é feita através da comparação do tipo devolvido pela expressão correspondente ao valor a ser atribuído e o tipo da variável à qual estamos a fazer a atribuição;
    
    \item \textbf{Erro 5} - Este erro é gerado quando, por exemplo, o tipo da expressão no \textit{return} é diferente do tipo de retorno do método. Para além disto, no caso de termos um \textit{if} ou um \textit{while} cuja expressão condicional não é do tipo boolean, é também gerado um erro semântico deste tipo;
    
    \item \textbf{Erro 6} - Nos casos em que é chamado o \textit{System.out.println} com váriaveis cujo tipo não é \textit{Integer} nem \textit{Boolean}, então é gerado um erro indicando que o operador apenas aceita parâmetros deste tipo.
    
    \item \textbf{Erro 7} - Caso o literal não seja um octal (começado por "0"), hexadecimal (começado por "0x") ou decimal (os restantes casos), é então gerado um erro deste tipo;
    
    \item \textbf{Erro 8} 
    
    \begin{itemize}
        \item Quando um operador unário ou o \textit{.length} é aplicado sobre um tipo sobre o qual essa operação não é valida. Por exemplo, o operador unário \textit{not}, apenas pode ser aplicado sobre expressões do tipo \textit{boolean}. Igualmente, o \textit{.length} apenas pode ser usado em \textit{arrays};
        
        \item Na inicialização de \textit{arrays}, caso a expressão que indica o tamanho do \textit{array} seja de um tipo diferente de \textit{integer}, é gerado um erro semântico.
    \end{itemize}
    
    
    \item \textbf{Erro 9}
    
    \begin{itemize}
    
        \item No caso de \textit{store array}, o erro pode ser análogo ao da indexação (Ver abaixo);
        
        \item Quando os operadores "+", "-", "*", "/", "\%", "<", ">", "<=", ">=" são aplicados a tipos diferentes de \textit{integer};
        
        \item Quando os operdores "!=" e "==" são aplicados a tipos diferentes;
         
        \item Quando os operadores "$\&\&$" e "$||$" são aplicados a tipos diferentes de \textit{Boolean} 
        
        \item Caso o \textit{Integer.parseInt} não seja aplicado a um \textit{array} de \textit{Strings} indexado por um \emph{integer}, então é gerado um erro deste tipo;
        
        \item É gerado um erro deste tipo quando é feita a indexação a um tipo diferente de \textit{int[]} ou \textit{bool[]}. Igualmente, caso a expressão de indexação não seja do tipo inteiro é gerado um erro semântico desta natureza;
    \end{itemize}
    
\end{itemize}

\section{Geração de Código}

A última fase, que se segue à fase semântica, é a geração de código. No âmbito do nosso projecto, foi implementada a geração de código intermédio \textit{LLVM}. Após a geração de um ficheiro \textit{.ll}, pode então interpretar-se com o comando \textit{lli} ou compilar com o comando \textit{llc} seguido de um qualquer compilador de \textit{Assembly} o código gerado pelo nosso compilador.

Nesta secção serão apenas mencionadas as funcionalidades mais complexas, para manter a brevidade deste relatório. As restantes funcionalidades, por serem comparativamente triviais, foram omitidas.

\subsection{Ifs e Whiles}

Para a implementação destas estruturas de controlo foram utilizadas \textit{named labels}, com o cuidado de usar pontos na sua nomenclatura, para evitar conflitos com identificadores do programa a ser compilado.

O \textit{if-else} consiste em 3 \textit{labels}, uma para o \textit{then}, uma para o \textit{else} e uma que representa o fim do \textit{if-else}. A \textit{then}, contém o código que deve ser executado se a condição for verdadeira, sendo que na \textit{else} se encontra o código que deve ser executado se esta for falsa. A \textit{label} que representa o fim do \textit{if-else} é para onde qualquer um dos segmentos anteriores salta no final da sua execução. Esta \textit{label} precede o código que se segue a esta estrutura de controlo de fluxo.

O \textit{while} é muito análogo ao \textit{if-else}, tendo também 3 \textit{labels}. A primeira é o início da estrutura, onde se encontra a condição de paragem. Caso esta condição seja verdadeira, o programa executa o que se encontra na segunda \textit{label}, \textit{do}, voltando no final deste segmento à \textit{label start}. Caso a condição de paragem seja falsa, a execução salta para a \textit{label end}, que precede o código que se segue ao \textit{while}.

\subsection{Returns}

Como não é efectuada qualquer análise da presença de \textit{returns} numa função, ou da existência de \textit{returns} em todos os ramos de execução de uma função, foi necessário encontrar uma forma de nos certificarmos que qualquer função retorna, independentemente da existência ou não de \textit{returns explícitos}.

Para solucionar este problema, todos os métodos possuem um return por omissão, colocado no final da função. Este retorna 0 no caso de o tipo de retorno ser \textit{inteiro} ou \textit{boolean}, \textit{null} no caso de ser um ponteiro ou uma estrutura com o primeiro campo a 0 e o segundo a \textit{null} no caso de ser um \textit{array}.

\subsection{ParseInt}

Para o \textit{Integer.parseInt}, foi usada a função \textit{atoi()} da biblioteca de \textit{C}, através da \textit{API} de chamada deste tipo de funções a partir de \textit{LLVM}.

\subsection{Prints}

Para a impressão foi necessário definir \textit{strings} auxiliares para serem usadas na função \textit{printf()} da biblioteca de \textit{C}.

No caso de inteiros, foi utilizada a \textit{string} \textit{"\%d\textbackslash n"}, sendo passada essa \textit{string} como \textit{format string} da função \textit{printf()} e o inteiro a imprimir como segundo argumento dessa mesma função.

No caso de \textit{booleans}, utilizou-se um \textit{array} com a string \textit{"false\textbackslash n"} na primeira posição e \textit{"true \textbackslash n"} na segunda. Desta forma, e tendo em conta que em \textit{LLVM} \textit{false} $= 0$ e \textit{true} $= 1$, podemos usar o valor que queremos imprimir como índice do \textit{array} para obter a \textit{format string} a passar à função \textit{printf()}.

\subsection{Short Circuiting}

Inicialmente tentámos implementar esta funcionalidade com recurso a nós \textit{phi} da representação intermédia \textit{LLVM}. No entanto, após problemas com a correcta indicação das \textit{labels} que precedem a instrução \textit{phi} desistimos desta abordagem.

No entanto, essa abordagem seria correcta e mais eficiente em termos de memórida alocada na stack, não fazendo uso de nenhuma memória deste tipo. Chegámos tardiamente à conclusão de que, com uma \textit{label} adicional que representasse sempre a saída de qualquer código que fosse executado apenas caso não existisse \textit{short circuiting}, poderiamos ter solucionado este problema.

A nossa abordagem implementada foi utilizar uma estrutura de controlo de fluxo semelhante a um \textit{if}, mas que necessita de recorrer a memória da pilha.

\lstset{caption={Short Circuiting em ANDs}, numbers=none}
\begin{lstlisting}
 a && b
 
 res = a;
 if(a)
    res = b;
\end{lstlisting}

\lstset{caption={Short Circuiting em ORs}, numbers=none}
\begin{lstlisting}
 a || b
 
 res = a;
 if(!a)
    res = b;
\end{lstlisting}

\subsection{Arrays e .length}

Para poder ter a funcionalidade de \textit{Java} existente no operador \textit{.length}, implementámos os \textit{arrays} como sendo estruturas, em que na primeira posição se encontra o tamanho do vector e na segunda o ponteiro em si.

Desta forma, o operador \textit{.length} consiste apenas na obtenção do valor guardado na primeira posição da estrutura. Este valor deve ser actualizado sempre que existir um \textit{new} com o valor correcto do novo tamanho do array, sendo inicializado a 0 aquando da declaração da variável.

A indexação de um array continua a ser uma simples indexação, em que apenas se deve indexar o valor na segunda posição da estrutura.

\subsection{Inicialização de Arrays}

Usando a função \textit{calloc} de \textit{C} forçamos a inicialização de \textit{arrays} de inteiros a 0 e de \textit{arrays} de \textit{booleans} a \textit{false}, tal como acontece em \textit{Java}.

\section{Possíveis Melhorias}

Uma das possíveis melhorias que poderiamos implementar seria a libertação de toda a memória alocada na \textit{heap} no processo de compilação de um qualquer programa. Apesar de termos o código já estruturado para que, em qualquer cenário de saída do programa, não tivémos, infelizmente, tempo para implementar esta correcta libertação.

Outra melhoria digna de ser mencionada é o facto de não utilizarmos \textit{unions} nos nós da \textit{AST} em casos em que o uso destas estruturas pouparia claramente espaço. A razão pela qual não implementámos desta forma os nós da \textit{AST} foi não complicar em demasia as funções de criação de nós da \textit{AST}, que nesse caso teriam de ter sido implementadas com cuidado para não sobrepor campos importantes ao inicializar campos irrelevantes para o tipo de nó em questão.

\end{document}
